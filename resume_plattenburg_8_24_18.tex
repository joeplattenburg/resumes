% LaTeX resume using res.cls
\documentclass[margin]{res}
%\usepackage{helvetica} % uses helvetica postscript font (download helvetica.sty)
%\usepackage{newcent}   % uses new century schoolbook postscript font 
\setlength{\textwidth}{5.1in} % set width of text portion

\begin{document}

% Center the name over the entire width of resume:
 \moveleft.5\hoffset\centerline{\large\bf Joseph Plattenburg} 
% Draw a horizontal line the whole width of resume:
 \moveleft\hoffset\vbox{\hrule width\resumewidth height 1pt}\smallskip
% address begins here
% Again, the address lines must be centered over entire width of resume:
 \moveleft.5\hoffset\centerline{226 Irving Way}
 \moveleft.5\hoffset\centerline{Columbus, OH 43214}
 \moveleft.5\hoffset\centerline{(937) 901-6576 | joeplattenburg@gmail.com}


\begin{resume}

 
\section{EXPERIENCE} 
                {\sl Director of Data Science} \hfill July 2018 | Present \\
                 Root Insurance Company, Columbus, OH
                 \begin{itemize}  \itemsep -2pt %reduce space between items
                 \item Led the telematics data science team in the implementation of                        new predictive scoring models 
                 \item Oversaw and comtributed to migration of production code to a 
                       more consistent and robust Python framework   
                 \end{itemize}
 
                {\sl Lead Data Scientist} \hfill Apr 2018 | July 2018 \\
                 Root Insurance Company, Columbus, OH
                 \begin{itemize}  \itemsep -2pt %reduce space between items
                 \item Built predictive models for scoring, distracted driving, and
                       driver passenger classification, improving predictive power by
                       nearly 2X
                 \end{itemize} 
                {\sl Advanced Engineer, R\&D} \hfill June 2016 | Apr 2017 \\
                 Owens Corning Science and Technology, Granville, OH
                 \begin{itemize}
                 \item Designed experimental procedure and data analysis algorithm
                       for material property testing
                 \item Led initiative for collaboration with university
                       researchers, leading to a funded project 
                 \end{itemize}
                {\sl Independent Consultant} \hfill 2015 | Present
                 \begin{itemize} \itemsep -2pt
                 \item Developed prototype software and hardware interface for 
                       real-time detection and classification of acoustic events
                 \end{itemize} 


\section{EDUCATION} 
               {\sl PhD}, Mechanical Engineering \\
                The Ohio State University, Columbus, OH \\
                May 2016, GPA: 4.0 \\
                Focus Areas: Acoustics/Vibrations. Signal Processing, Modeling

                {\sl Bachelor of Science}, Mechanical Engineering \\
                The Ohio State University, Columbus, OH \\ 
                June 2012, GPA: 3.97 \\ 
                Minors: Mathematics, Music 
 
 
\section{SKILLS} 
               {\sl Operating Systems:} Ubuntu, Windows, Mac \\ 
               {\sl Languages:} Python, R, MATLAB, SQL (Postgres/Redshift), C, bash\\
               {\sl Modeling/Analysis:} GLMs, GBMs, random forests, neural networks, 
                    PCA, time/frequency domain methods (FFT, autocorrelation, 
                    wavelets) \\
               {\sl Relevant Coursework:} Digital Signal Processing, Advanced Linear 
                    Algebra/Linear System Theory, Numerical Methods, Statistics \\ 
               {\sl Spanish Language:} read, write, and speak with basic competence


\section{RESEARCH}
               {\sl PhD Dissertation:} Analytical Vibration Models for Plates and
                    Shells with Combined Active and Passive Damping
               \begin{itemize}
               \item Developed semi-analytical models of structural noise and
                     vibration response with experimental validation
               \end{itemize}
               {\sl Undergraduate Research:} Bearing Health and Load Monitoring Study
               \begin{itemize}
               \item Measured frequency response of automotive bearing for
                     diagnostics and failure prediction
               \end{itemize}


\end{resume}

\end{document}




